% Copyright (c) Lars Niedorf, Jan Path 2018
%
% This work is licensed under the Creative Commons Attribution-ShareAlike 4.0
% International License. To view a copy of this license, visit
% http://creativecommons.org/licenses/by-sa/4.0/ or send a letter to Creative
% Commons, PO Box 1866, Mountain View, CA 94042, USA.

% !TEX root = main.tex

% 18-01-2018

%--------------------------------------------------------------------------------------------------------------------

\section{Triangular matrix rings}

%--------------------------------------------------------------------------------------------------------------------

Let $\Lambda$ and $\Gamma$ be two arbitrary rings. Let $M$ be a $(\Lambda,\Gamma)$-bimodule. Given these data, we define a new ring
\[
T :=
\begin{pmatrix}
\Lambda & M\\
0 & \Gamma
\end{pmatrix},
\]
whose elements will be written as
$
\left(
\begin{smallmatrix}
\lambda & m\\
0 & \gamma
\end{smallmatrix}
\right),
$
where $\lambda\in\Lambda$, $m\in M$ and $\gamma\in\Gamma$. The addition is defined componentwise and the multiplication is defined via
\[
\begin{pmatrix}
\lambda & m\\
0 & \gamma
\end{pmatrix}
\begin{pmatrix}
\lambda' & m'\\
0 & \gamma'
\end{pmatrix}
:=
\begin{pmatrix}
\lambda\lambda' & \lambda m' + m \gamma'\\
0 & \gamma\gamma'
\end{pmatrix}.
\]
%
Then $T$ is a ring. Moreover if $\Lambda$ and $\Gamma$ are $R$-algebras over some commutative ring $R$ and $M$ is a bimodule for these algebras, that is $(r\cdot 1_\Lambda).m=m.(1_\Gamma \cdot r)$ for all $r\in R$, then $T$ is an $R$-algebra.

%--------------------------------------------------------------------------------------------------------------------

\begin{lemma}
Suppose that $\Lambda$ and $\Gamma$ are artin $R$-algebras. If $M$ is finitely generated over $R$, then
$
T =
\left(\begin{smallmatrix}
\Lambda & M\\
0 & \Gamma
\end{smallmatrix} \right)
$
is an artin $R$-algebra.
\end{lemma}

%--------------------------------------------------------------------------------------------------------------------

\begin{proof}
As an $R$-module we have
$
\left(\begin{smallmatrix}
\Lambda & M\\
0 & \Gamma
\end{smallmatrix} \right)
\cong
\Lambda\oplus\Gamma\oplus M
$.
\end{proof}

%--------------------------------------------------------------------------------------------------------------------

Let $\Lambda$ and $\Gamma$ be artin $R$-algebras. Let $M$ be a $(\Lambda,\Gamma)$-bimodule that is finitely generated over $R$. We consider the artin $R$-algebra
$
T =
\left(\begin{smallmatrix}
\Lambda & M\\
0 & \Gamma
\end{smallmatrix} \right)
$
as before as well as the category $\mathcal{C}_T$:
\begin{enumerate}[label=(\roman*)]
\item The objects of $\mathcal{C}_T$ are triples $(X,Y,f)$, where $X\in\mod\Lambda$, $Y\in\mod\Gamma$ and
\[
f:M\otimes_\Gamma Y \to X
\]
is $\Lambda$-linear.
\item A morphism $\varphi:(X,Y,f)\to (X',Y',f')$ is a pair $(\alpha,\beta)$ of morphism $\alpha:X\to X'$ and $\beta:Y\to Y'$ such that the diagram
\[\begin{tikzcd}
      M \otimes_\Gamma Y \dar["\id_M \otimes \beta"] \rar["f"] & X \ar[d, "\alpha"]\\
      M \otimes_\Gamma Y' \rar["f'"] & X'
          \end{tikzcd}
\]
commutes. Sums of morphisms are defined componentwise.
\end{enumerate}
%
With this data, $\mathcal{C}_T$ is an $R$-category.

We define a functor $F:\mathcal{C}_T\to\mod T$ via $F(X,Y,f):=X\oplus Y$, where $X\oplus Y$ is a direct sum of $R$-modules with $T$ acting on it via
\[
\begin{pmatrix}
\lambda & m\\
0 & \gamma
\end{pmatrix}
\begin{pmatrix}
x \\ y
\end{pmatrix}
=
\begin{pmatrix}
\lambda x + f(m\otimes y) \\ \gamma y
\end{pmatrix}.
\]
Moreover we define $F(\alpha,\beta):=\alpha\oplus\beta$.

%--------------------------------------------------------------------------------------------------------------------

\begin{proposition}\label{3.4.2}
The functor $F:\mathcal{C}_T\to\mod T$ is an equivalence of $R$-categories.
\end{proposition}

%--------------------------------------------------------------------------------------------------------------------

\begin{proof}
We first show that $F$ is full. A morphism $\omega:F(X,Y,f)\to F(X',Y',f')$ is a $T$-linear map $\omega:X\oplus Y\to X'\oplus Y'$. Hence there is an $R$-linear map which we write as a matrix
\[
\omega=\begin{pmatrix}
\alpha & \alpha' \\
\beta' & \beta
\end{pmatrix}
\]
of $R$-linear maps. Given $\lambda\in\Lambda$, $\gamma\in\Gamma$ and $m\in M$ we have
\begin{align*}
\begin{pmatrix}
\lambda & m \\
0 & \gamma
\end{pmatrix}
\cdot
\omega
\begin{pmatrix}
x \\ y
\end{pmatrix}
& =
\begin{pmatrix}
\lambda & m \\
0 & \gamma
\end{pmatrix}
\begin{pmatrix}
\alpha(x) + \alpha'(y) \\ \beta'(x) + \beta(y)
\end{pmatrix} \\
& =
\begin{pmatrix}
\lambda\alpha(x) + \lambda \alpha'(y) + f'(m\otimes \beta'(x)) + f'(m\otimes \beta(y)) \\
\gamma\beta'(x) + \gamma\beta(y)
\end{pmatrix}
\end{align*}
%
while
\begin{align*}
\omega\left(
\begin{pmatrix}
\lambda & m \\
0 & \gamma
\end{pmatrix}
\begin{pmatrix}
x \\ y
\end{pmatrix}
\right)
& =
\omega\begin{pmatrix}
\lambda x + f(m\otimes y) \\ \gamma y
\end{pmatrix} \\
& =
\begin{pmatrix}
\alpha(\lambda x) + \alpha(f(m\otimes y)) + \alpha'(\gamma y) \\
\beta'(\lambda x) + \beta'(f(m\otimes y)) + \beta(\gamma y)
\end{pmatrix}
\end{align*}
%
Setting $m=0$ and $y=0$ implies that $\alpha$ is $\Lambda$-linear. Moreover we obtain that $\beta$ is $\Gamma$-linear by setting $m=0$ and $x=0$. For $m=0$ and $y=0$ the second line of the matrix now reads $\gamma\beta'(x)=\beta'(\lambda x)$ for all $\lambda$, $\gamma$ and $x$ so that $\beta'=0$. Similarly, $\alpha'=0$. Consequently, $f'(m\otimes \beta(y))=\alpha(f(m\otimes y))$. As a result
\[
\omega = F(\alpha,\beta).
\]
It follows that $F$ is full. Moreover $F$ is faithful since $F(\alpha,\beta)=0$ implies $(\begin{smallmatrix} \alpha & 0 \\ 0 & \beta \end{smallmatrix})=\alpha\oplus\beta = 0$ and $\alpha=\beta=0$. We now show that $F$ is dense. Let $V\in\mod T$. We define $e_1=(\begin{smallmatrix} 1 & 0 \\ 0 & 0 \end{smallmatrix})$ and $e_2=(\begin{smallmatrix} 0 & 0 \\ 0 & 1 \end{smallmatrix})$. There results a decomposition $V=e_1V\oplus e_2V$ of $R$-modules. Let $X:=e_1 V$ and $Y:=e_2 V$. Setting
\[
\lambda.(e_1v) := \begin{pmatrix} \lambda & 0 \\ 0 & 0 \end{pmatrix} e_1 v
\qq{and}
\gamma.(e_2v) := \begin{pmatrix} 0 & 0 \\ 0 & \gamma \end{pmatrix} e_2 v,
\]
the spaces $X$ and $Y$ obtain the structure of a $\Lambda$-module and a $\Gamma$-module, respectively. We consider the map
\[
\widetilde f:
\declaremap{M\times e_2V}{e_1V}{(m,e_2v)}{(\begin{smallmatrix} 0 & m \\ 0 & 0 \end{smallmatrix})v}.
\]
If $e_2v=0$, then $(\begin{smallmatrix} 0 & m \\ 0 & 0 \end{smallmatrix}) v = (\begin{smallmatrix} 0 & m \\ 0 & 0 \end{smallmatrix}) e_2v = 0$. Since $e_1 (\begin{smallmatrix} 0 & m \\ 0 & 0 \end{smallmatrix})=(\begin{smallmatrix} 0 & m \\ 0 & 0 \end{smallmatrix})$, we have
\[
\widetilde{f}(M\times e_2V)\subseteq e_1V.
\]
Note that $\widetilde f$ is $\Z$-bilinear and $\Gamma$-balanced since
\[
\widetilde{f}(m\gamma,e_2v)
 = \begin{pmatrix} 0 & m\gamma \\ 0 & 0 \end{pmatrix} v
 = \begin{pmatrix} 0 & m \\ 0 & 0 \end{pmatrix} \gamma e_2v
 = \widetilde{f}(m,\gamma e_2v).
\]
Hence there is $f:M\otimes_\Gamma e_2V\to e_1V$ such that $f(m\otimes e_2v) = (\begin{smallmatrix} 0 & m \\ 0 & 0 \end{smallmatrix})v$. Direct computation shows that $f$ is $\Lambda$-linear. Moreover we have $F(X,Y,f)\cong V$. By Theorem~\ref{3.1.1}, $F$ is an equivalence.
\end{proof}

%--------------------------------------------------------------------------------------------------------------------

