% Copyright (c) Lars Niedorf, Jan Path 2018
%
% This work is licensed under the Creative Commons Attribution-ShareAlike 4.0
% International License. To view a copy of this license, visit
% http://creativecommons.org/licenses/by-sa/4.0/ or send a letter to Creative
% Commons, PO Box 1866, Mountain View, CA 94042, USA.

% !TEX root = main.tex

% 08-01-2018

%--------------------------------------------------------------------------------------------------------------------

%Setting: $\Lambda$ is an artinian $R$-algebra. $M \in \mod{\Lambda}; \add{M} \subseteq \mod{\Lambda}$. $\Gamma := \End_\Lambda(M)^\op$, $e_M := \left\{ \begin{matrix}
%    \mod{\Lambda} &\to \mod{\Gamma}\\
%    N &\mapsto \Hom_\Lambda(M,N)
%  \end{matrix} \right.$

%\begin{proposition}\label{3.2.2}
%  Let $M \in \mod{\Lambda}$, $\Gamma := \End_\Lambda(M)^\op$.
%  \begin{enumerate}
%  \item If $X \in \add{M}$ and $Y \in \mod{\Lambda}$, then
%    \[ e_M : \Hom_\Lambda(X,Y) \to \Hom_\Gamma(e_M(X),e_M(Y)) \]
%    is an isomorphism.
%  \item If $X \in \add{M}$, then $e_M(X) \in \Projectives(\Gamma)$.
%  \item $\restr{e_M}{\add{M}} : \add{M} \to \Projectives(\Gamma)$ is an equivalence of
%    categories.
%  \end{enumerate}
%\end{proposition}

%Let $M \in \mod{\Lambda}$, then an exact sequence
% \[\begin{tikzcd}
%     P_1 \rar& P_0 \rar& M \rar& (0)
%   \end{tikzcd}\]
%is called a projective presentation.


%Given a projective module P, we let $\mod{P}$ be the full category of $\mod{\Lambda}$
%where each object $M \in \mod{P}$ has a projective presentation $P_1 \to P_0 \to M
%\to (0)$ where $P_i \in \add{P}$.

\begin{proposition}\label{3.2.3}
  Let $P \in \mod{\Lambda}$ be projective and $\Gamma := \End_\Lambda(P)^\op$. Then
  \[
  e_P : \mod{P} \to \mod{\Gamma}
  \]
is an equivalence of $R$-categories.
\end{proposition}

%--------------------------------------------------------------------------------------------------------------------

\begin{proof}
  Let $X \in \mod{\Gamma}$ be a $\Gamma$-module and
   \[\begin{tikzcd}
    Q_1 \rar["g"] & Q_0 \rar& X \rar& (0)
   \end{tikzcd}\]
be a projective representation of $X$. By virtue of Proposition~\ref{3.2.2}, we can find $P_i \in \add{P}$, where $0 \leq i \leq 1$, such that
  $Q_i \isomorphic e_P(P_i)$. By the same token, there is a morphism $f: P_1 \to P_0$, such
  that the diagram
  \[\begin{tikzcd}
      e_P(P_1) \rar["e_P(f)"] \dar["\wr"] & e_P(P_0) \dar["\wr"]\\
      Q_1 \rar["g"]& Q_0
    \end{tikzcd}\]
commutes. Since $P$ is projective, the functor $e_P$ is exact.
  Consequently,
    \[ X = \coker g \isomorphic \coker (e_P(f)) \isomorphic e_P(\coker f).\]
  By construction $\coker f \in \mod{P}$. Hence $e_P$ is dense.
  Now let $M,N \in \mod{P}$, and let $P_1 \to P_0 \to M \to (0)$ be a representation of $M$
  such that $P_i \in \add{P}$. There results a commutative diagram
  \[\begin{tikzcd}[column sep=small]
      (0) \rar& \Hom_\Lambda(M,N) \dar["e_P"]\rar& \Hom_\Lambda(P_0,N) \dar["e_P"]\rar& \Hom_\Lambda(P_1,N) \dar["e_P"]\\
      (0) \rar& \Hom_\Gamma(e_P(M),e_P(N)) \rar& \Hom_\Gamma(e_P(P)_0,e_P(N)) \rar& \Hom_\Gamma(e_P(P)_1,e_P(N))
    \end{tikzcd}\]
  with exact rows.
By Proposition~\ref{3.2.2}, the second and third vertical arrows are isomorphisms. Hence so
is the first left-hand arrow.
As a result, $e_P$ is full and faithful.
\end{proof}

%--------------------------------------------------------------------------------------------------------------------

\begin{definition}
  A projective $\Lambda$-module $P \in \mod{\Lambda}$ is referred to as a \textbf{projective generator}\index{projective
    generator} if for every $M \in \mod{\Lambda}$, there is $n \in \mathbb{N}$ and a surjection $P^n \surjection M$.
\end{definition}

%--------------------------------------------------------------------------------------------------------------------

\begin{example}
  $\Lambda \in \mod{\Lambda}$ is a projective generator (simply because $M \in \mod{\Lambda}$ is
  finitely generated).
\end{example}

%--------------------------------------------------------------------------------------------------------------------

\begin{theorem}[Morita]\label{3.2.4}
  Let $\Lambda$ and $\Gamma$ be artin $R$-algebras. Then the following statements are
  equivalent:
  \begin{enumerate}
  \item There exists an equivalence $F : \mod{\Lambda} \to \mod{\Gamma}$ of $R$-categories.
  \item There exists a projective generator $P \in \mod{\Lambda}$, such that $\Gamma
    \isomorphic \End_\Lambda(P)^\op$.
  \end{enumerate}
\end{theorem}

%--------------------------------------------------------------------------------------------------------------------

\begin{proof}\
\begin{addmargin}[1cm]{0cm}
\hspace{-1cm}(1) $\Rightarrow$ (2): Suppose that $F : \mod{\Lambda} \to \mod{\Gamma}$ is an equivalence. Owing to Proposition~\ref{3.1.2}, the module $P := F^{-1}(\Gamma)$ is projective.
  If $X \in \mod{\Lambda}$, then there is $M \in \mod{\Gamma}$ with $X \isomorphic F^{-1}(M)$. As $M$ is
  finitely generated, there is $n \in \mathbb{N}$ and an epimorphism
  $f: \Gamma^n \twoheadrightarrow M$. In view of Proposition~\ref{3.1.2},
    \[ F^{-1}(f) : F^{-1}(\Gamma^n) \to F^{-1}(M) \]
  is also an epimorphism. As $F^{-1}$ interchanges with finite direct sums, there
  results an epimorphism $P^n \surjection X$.
  Consequently, $P$ is a projective generator of $\mod{\Lambda}$. Moreover
    \[ \End_\Lambda(P)^\op \overset{f}\isomorphic \End_\Gamma(\Gamma)^\op \isomorphic \Gamma. \]
%
\hspace{-1cm}(2) $\Rightarrow$ (1): Since $P$ is a projective generator, we have $\add(P) =
  \Projectives(\Lambda)$. Hence we obtain $\mod{P} = \mod{\Lambda}$. The assertion now follows from Proposition~\ref{3.2.3}.\qedhere
\end{addmargin}
\end{proof}

%--------------------------------------------------------------------------------------------------------------------

\begin{remark}
  Let $F : \mod{\Lambda} \to \mod{\Gamma}$ be an $R$-functor. Then $F(M \oplus N) \isomorphic F(M) \oplus F(N)$. We have that the exact sequence
  \[
  \begin{tikzcd}
    (0) \rar& N \rar["\iota"]& M \oplus N \rar["\pi"]& M \lar["\lambda", bend left] \rar& (0)
  \end{tikzcd}
  \]
  splits. Then $F(\pi) \circ F(\lambda) = F(\pi \circ \lambda) = F(\id_M) = \id_{F(M)}$ which implies that
  \[
  \begin{tikzcd}
  (0) \rar& F(N) \rar& F(M \oplus N) \rar& F(M) \rar& (0)
  \end{tikzcd}
  \]
  is split exact. Hence $F(M \oplus N) \isomorphic F(M) \oplus F(N)$.
\end{remark}

%--------------------------------------------------------------------------------------------------------------------

\begin{corollary}\label{3.2.5}
  Let $P_1, \ldots, P_n$ be a complete set of representatives for the isomorphism
  classes of the principal indecomposable $\Lambda$-module. Then
  \[
  e_{\bigoplus_{i=1}^n P_i} :
  \mod{\Lambda} \to \mod{\End_\Lambda{\left(\bigoplus_{i=1}^n P_i\right)}^\op}
  \] is an equivalence.
\end{corollary}

%--------------------------------------------------------------------------------------------------------------------

\begin{example}
  Let $V$ be a finite-dimensional $k$-vector space. Let $\Lambda := \End_k(V)$. Then $n = 1$ and $P_1 \isomorphic V$.
  Hence $\Lambda$ is Morita equivalent to $\End_\Lambda(V) = k$.
\end{example}

%--------------------------------------------------------------------------------------------------------------------

\begin{proof}
  Let $M \in \mod{\Lambda}$. Then there exists an epimorphism $n\Lambda \surjection M$. Recall that $\Lambda \isomorphic
  \bigoplus_{i=1}^n m_i P_i$. Let $m_0 := \max\{m_i \mid 1 \leq i \leq n\}$. Then there is an
  epimorphism
  \[
  m_0\left(\bigoplus_{i=1}^n P_i\right) \surjection m \Lambda \surjection M.
  \]
  Hence $\bigoplus_{i=1}^n P_i$ is a projective generator.
\end{proof}

%--------------------------------------------------------------------------------------------------------------------

\begin{definition}
  An artin $R$-algebra is referred to as being \textbf{basic}\index{basic algebra} if $\Lambda \isomorphic \bigoplus_{i=1}^n
  P_i$, where $P_i$ is principal indecomposable with $P_i \not\isomorphic P_j$ for $i \neq j$.
\end{definition}

%--------------------------------------------------------------------------------------------------------------------

\begin{corollary}\label{3.2.6}
  Let $\Lambda$ and $\Gamma$ be two basic artin $R$-algebras. Then the following statements
  are equivalent:
  \begin{enumerate}
  \item The categories $\mod{\Lambda}$ and $\mod{\Gamma}$ are equivalent.
  \item The algebras $\Lambda$ and $\Gamma$ are isomorphic.
  \end{enumerate}
\end{corollary}
\begin{proof} We show (1) $\Rightarrow$ (2). Let $F : \mod{\Lambda} \to \mod{\Gamma}$ be an equivalence. According to the
  proof of Theorem~\ref{3.2.4}, $P := F^{-1}(\Gamma)$ is a projective generator for $\Lambda$ with
  $F \isomorphic \End_\Lambda(P)^\op$.
  Since $\Gamma$ is basic, we have
    \[ \Gamma \isomorphic \bigoplus_{i=1}^n Q_i \]
  with $Q_i$ principal indecomposable and $Q_i \not\isomorphic Q_j$ for $i \neq j$.
  Thus writing $P_i := F^{-1}(Q_i)$ we have
  \[
  P \isomorphic \bigoplus_{i=1}^n P_i
  \]
  with $P_i \not\isomorphic P_j$ for
  $i \neq j$. Let $P'$ be a principal indecomposable $\Lambda$-module. Then $F(P')$ is
  projective and indecomposable.
  Hence $F(P') \isomorphic Q_i$ for some $i \in \{1, \ldots, n\}$. Thus $P' \isomorphic F^{-1}(Q_i) = P_i$.
  Since $\Lambda$ is basic, we obtain $\Lambda \isomorphic \bigoplus_{i=1}^n P_i \isomorphic P$. Hence $\Gamma
  \isomorphic \End_\Lambda(P)^\op \isomorphic \End_\Lambda(\Lambda)^\op \isomorphic \Lambda$.
\end{proof}

%--------------------------------------------------------------------------------------------------------------------
