% Copyright (c) Lars Niedorf, Jan Path 2018
%
% This work is licensed under the Creative Commons Attribution-ShareAlike 4.0
% International License. To view a copy of this license, visit
% http://creativecommons.org/licenses/by-sa/4.0/ or send a letter to Creative
% Commons, PO Box 1866, Mountain View, CA 94042, USA.

% !TEX root = main.tex

% 22-01-2018

%--------------------------------------------------------------------------------------------------------------------

\begin{corollary}\label{3.4.3}
Let $X \in \mathcal{C}_T$ be an indecomposable projective object. Then there either exists
an indecomposable projective $\Gamma$-module Q with $X \isomorphic (M \otimes_\Gamma Q, Q, \id_{M \otimes_\Gamma
Q})$or there exists an indecomposable projective $\Lambda$-module P such that $X
\isomorphic (P, 0, 0)$.
\end{corollary}

%--------------------------------------------------------------------------------------------------------------------

\begin{proof}
Let $(X,Y,f) \in \mathcal{C}_T$ be an indecomposable projective object. According to
Proposition~\ref{3.4.2} and Proposition~\ref{3.1.2}, $F(X,Y,f)$ is an indecomposable projective $T$-module.
Note that $T = T_1 \oplus T_2$ with
\[
T_1 =
\begin{pmatrix}
\Lambda & 0 \\ 0 & 0
\end{pmatrix} \qq{and}
T_2 =
\begin{pmatrix}
0 & M \\ 0 & \Gamma
\end{pmatrix}
\]
is a decomposition of the regular module. Since $e_1 T_1 = \Lambda$, $e_1 T_2 = (0)$ and $e_2 T_1 = (0)$, it follows from the proof of Proposition~\ref{3.4.2} that $T_1 \isomorphic F(\Lambda, 0, 0)$. Note that
\[
g: \declaremap{M \otimes_\Gamma \Gamma}{M}{m \otimes \gamma}{m\gamma}
\]
is an isomorphism of $\Lambda$-modules. We therefore have
\[
T_2 = F(M, \Gamma, g) \isomorphic F(M \otimes_\Gamma \Gamma, \Gamma, \id_{M \otimes_\Gamma \Gamma}).
\]
Since $F(X, Y, f)$ is a principal indecomposable $T$-module it is isomorphic to a direct summand of $T_1$ or $T_2$ by the theorem of Krull-Remak-Schmidt. In the former case, $(X,Y,f) \isomorphic (P,0,0)$ with $P$ principal indecomposable $\Lambda$-module. In the latter case, there is a principal indecomposable $\Gamma$-module Q such that $(X,Y,f) \isomorphic (M \otimes_\Gamma Q, Q, \id_{M \otimes_\Gamma Q})$.
\end{proof}

%--------------------------------------------------------------------------------------------------------------------

\section{Algebras of radical square zero}\label{3.5}

%--------------------------------------------------------------------------------------------------------------------

In this section let $\Lambda$ be an artin $R$-algebra with Jacobson radical $J=J(\Lambda)$. If $J = (0)$, then Wedderburn's Theorem
tells us
\[
\Lambda \isomorphic \bigoplus_{i=1}^n \Mat_{n_i}(\Delta_i),
\]
where the $\Delta_i$ are division rings. In this section, we are interested in the case $J^2 = (0)$. Let $T_\Lambda$ be the triangular matrix algebra (over $R$) given by
\[
T_\Lambda
=
\begin{pmatrix}
\factor \Lambda {J} & J \\
0 & \factor \Lambda {J}
\end{pmatrix}.
\]

%--------------------------------------------------------------------------------------------------------------------

\begin{definition}
An $R$-algebra is \textbf{hereditary}\index{hereditary algebra} provided all left ideals of $\Lambda$ are projective.
\end{definition}

%--------------------------------------------------------------------------------------------------------------------

\begin{remark}
Basic homological algebra yields that the following statements are equivalent for $\Lambda$ being an artin $R$-algebra:
\begin{enumerate}[label=(\roman*)]
\item $\Lambda$ is hereditary.
\item $J$ is a projective left $\Lambda$-module.
\item $\Ext_\Lambda^2(M,N) = (0)$ for all $M, N \in \mod \Lambda$.
\end{enumerate}
\end{remark}

%--------------------------------------------------------------------------------------------------------------------

\begin{lemma}\label{3.5.1}
Suppose that $J^2 = (0)$. Then $T_\Lambda$ is hereditary.
\end{lemma}

%--------------------------------------------------------------------------------------------------------------------

\begin{proof}
It suffices to show that $J_T := \Rad(T_\Lambda)$ is projective. Let $M := J$ and $\Lambda' := \factor \Lambda {J}$. Then $T_\Lambda =
(\begin{smallmatrix}
\Lambda' & M \\
0 & \Lambda'
\end{smallmatrix})
$. In particular, $ M \isomorphic
(\begin{smallmatrix}
0 & M\\
0 & 0
\end{smallmatrix})$ is an ideal of $T_\Lambda$ such that
\[
\factor {T_\Lambda} M \isomorphic \Lambda' \times \Lambda'.
\]
Hence $\factor {T_\Lambda} M$ is semi-simple, so that $J_T \subseteq M$. On the other hand, we have $M \subseteq J_T$ as $M^2 = (0)$. Since $M$ corresponds to $(M, 0, 0)$ and $M$ is a projective $\Lambda'$-module it follows from Corollary~\ref{3.4.3} that $M$ is projective. Hence $T_\Lambda$ is hereditary.
\end{proof}

%--------------------------------------------------------------------------------------------------------------------

\begin{lemma*}
Let $M \in \mod \Lambda$.
\begin{enumerate}
\item The map \[\mu: \declaremap{\Lambda \otimes_\Lambda M}{M}{\lambda \otimes m}{\lambda m}\] is an isomorphism of
abelian groups.
\item Let $X$ be a left $\Lambda$-module Then $- \otimes_\Lambda X: \Mod^r \Lambda \to \Mod{\mathbb{Z}}$ is right exact.
\item If $X$ above is projective, then $- \otimes_\Lambda X$ is exact.
\end{enumerate}
\end{lemma*}

%--------------------------------------------------------------------------------------------------------------------

We omit the proof.

%--------------------------------------------------------------------------------------------------------------------

\begin{lemma*}
Let $P$ be a projective $\Lambda$-module. Then $\mu$ restricts to an isomorphism \[I
\otimes_\Lambda P \xrightarrow{\sim} IP\] for every ideal $I \trianglelefteq \Lambda$.
\end{lemma*}

%--------------------------------------------------------------------------------------------------------------------

\begin{proof}
We consider the exact sequence
\[\begin{tikzcd}
(0) \rar& I \rar["\iota"]& \Lambda \rar["\pi"]& \factor \Lambda I \rar& (0)
\end{tikzcd}.\]
Since $P$ is projective, we obtain a commutative diagram
\[\begin{tikzcd}
(0) \rar & I \otimes_\Lambda P \dar["\mu|_{I \otimes_\Lambda P}"] \rar["\iota \otimes \id_P"] & \Lambda \otimes_\Lambda P
\dar["\wr"',"\mu"] \rar["\pi \otimes \id_P"] & \factor \Lambda I \otimes_\Lambda P \dar["\overline \lambda \otimes p
\mapsto \overline{\lambda p}"] \rar & (0) \\
(0) \rar & IP \rar & P \rar & \factor P {IP} \rar & (0)
\end{tikzcd}
\]
with exact rows.
\end{proof}

%--------------------------------------------------------------------------------------------------------------------

Suppose that $J^2 = (0)$. We are going to define a functor
\[ F: \mod \Lambda \to \mathcal{C}_{T_\Lambda}.\]
The idea is that $M \in \mod \Lambda$ is “determined” by its Loewy layers $\factor M {J M}$ and $J M$.
Let $M \in \mod \Lambda$. The multiplication $\Lambda \otimes_\Lambda M \to M$ induces a map
\[
J \otimes_{\factor \Lambda {J}} \factor M {J M} \to J M.
\]
The map
\[
\declaremap{J \times \factor M {J M}}{J M}{(x, \overline m)}{x.m}
\]
is well-defined, $\mathbb{Z}$-bilinear and $\factor \Lambda {J}$-balanced. Hence there is a $\mathbb{Z}$-linear map
\[
f: \declaremap{J \otimes_{\factor \Lambda {JM}}
\factor M {J}}{J M}{x \otimes \overline m}{x.m}
\]
This map is even $R$-linear, and $\factor \Lambda {J}$-linear. If $M \in \mod \Lambda$, we
put
\[
F(M) := (J M, \factor M {J M}, f) \in \mathcal{C}_{T_\Lambda}.
\]
Let $\varphi : M \to N$ be $\Lambda$-linear. Then $\varphi(J M) \subseteq \J{\varphi(M)} \subseteq J N$. There results a map
\[
\varphi_2: \declaremap{\factor M {J M}}{\factor N {J N}}{m + J M}{\varphi(m) + J N}
\]
which is $\factor \Lambda {J}$-linear. We put $\varphi_1 := \restr \varphi {J M} : J M
\to J N$ and define
\[
F(\varphi) := (\varphi_1, \varphi_2) \in \Hom_{\mathcal{C}_{T_\Lambda}}(F(M), F(N)).
\]
Direct computation shows that $F$ is a functor of $R$-categories.

%--------------------------------------------------------------------------------------------------------------------
