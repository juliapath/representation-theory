% Copyright (c) Lars Niedorf, Jan Path 2018
%
% This work is licensed under the Creative Commons Attribution-ShareAlike 4.0
% International License. To view a copy of this license, visit
% http://creativecommons.org/licenses/by-sa/4.0/ or send a letter to Creative
% Commons, PO Box 1866, Mountain View, CA 94042, USA.

% !TEX root = main.tex

% 23-10-2017

%--------------------------------------------------------------------------------------------------------------------

% CHAPTER 1

\chapter{Basic concepts}

%--------------------------------------------------------------------------------------------------------------------

% SECTION 1.1

\section{Artinian and noetherian rings and modules}

%--------------------------------------------------------------------------------------------------------------------

Throughout, $\Lambda$ denotes a ring (with identity). Let $\Mod \Lambda$ be the category of left $\Lambda$-modules. The proofs of this section are left as exercises to the reader.

%--------------------------------------------------------------------------------------------------------------------

\begin{definition}
Let $M$ be a $\Lambda$-module.
\begin{enumerate}
\item $M$ is \textbf{simple}\index{simple module} if $M\neq (0)$ and if $(0)$ and $M$ are the only submodules of $M$.
\item $M$ is \textbf{indecomposable}\index{indecomposable module} if $M\neq (0)$ and if $(0)$ and $M$ are the only direct summands of $M$.
\end{enumerate}
\end{definition}

%--------------------------------------------------------------------------------------------------------------------

\begin{example}
Let $k$ be a field and $X$ be an indeterminate.
\begin{enumerate}
\item If $\Lambda= k$, then any $\Lambda$-module $M$ is simple if and only if $\dim_k M = 1$.
\item Let $\Lambda=k[X]/(X^n)$ with $n\ge 2$.  The $\Lambda$ has exactly one simple module $S=k$, ($X.1=0$), while $M_i:=k[X]/(X^i)$ is the complete list of indecomposable $\Lambda$-modules. The indecomposable modules correspond to the Jordan canonical forms of nilpotent matrices of nilpotency class at most $n$.   
\end{enumerate} 
\end{example}

%--------------------------------------------------------------------------------------------------------------------

\begin{definition}
Let $M$ be a $\Lambda$-module.
\begin{enumerate}
\item $M$ is \textbf{noetherian}\index{noetherian module} if every ascending sequence $(M_n)_{n\ge 1}$ of submodules of $M$ is stationary.
\item $M$ is \textbf{artinian}\index{artinian module} if every descending sequence $(M_n)_{n\ge 1}$ of submodules of $M$ is stationary.
\end{enumerate}
\end{definition}

%--------------------------------------------------------------------------------------------------------------------

\begin{example}\
\begin{enumerate}
\item If $\Lambda=k$, then noetherian modules (artinian modules) are the finite dimensional vector spaces.
\item If $\Lambda=\Z$, then the regular module $M:=\Z$ is noetherian, but not artinian.
\item If $M$ is noetherian (artinian), so are its submodules and factor modules.
\end{enumerate}
\end{example}

%--------------------------------------------------------------------------------------------------------------------

\begin{proposition}\label{1.1.1}
Let $M$ be a $\Lambda$-module. Then the following statements are equivalent:
\begin{enumerate}
\item $M$ is noetherian.
\item Every non-empty set of submodules of $M$ possesses a maximal element.
\item Every submodule of $M$ is finitely generated.
\end{enumerate}
\end{proposition}

%--------------------------------------------------------------------------------------------------------------------

\begin{proposition}\label{1.1.2}
Let
\[
\begin{tikzcd}
	(0) \ar[r] & M' \ar[r] & M \ar[r] & M'' \ar[r] & (0)
\end{tikzcd}
\]
be an exact sequence of $\Lambda$-modules. If $M'$ and $M''$ are noetherian, so is $M$.
\end{proposition}

%--------------------------------------------------------------------------------------------------------------------

\begin{definition}
Let $M$ be a $\Lambda$-module. A finite sequence
\[
(0)=M_0\subseteq M_1\subseteq \dots \subseteq M_n = M
\]
of $\Lambda$-submodules of $M$ is called a \textbf{composition series}\index{composition series} of $M$ provided $M_i/M_{i-1}$ is simple for all $1\leq i\leq n$.
\end{definition}

%--------------------------------------------------------------------------------------------------------------------

\begin{lemma}\label{1.1.3}
A $\Lambda$-module $M$ has a composition series if and only if $M$ is artinian and noetherian.
\end{lemma}

%--------------------------------------------------------------------------------------------------------------------

\begin{theorem}[Jordan-Hölder]\label{1.1.4}
Let $M$ be a $\Lambda$-module.
Suppose that
\[
(0)=M_0\subseteq M_1\subseteq \dots \subseteq M_n = M
\qq{and}
(0)=M_0'\subseteq M_1'\subseteq \dots \subseteq M_m' = M
\]
are two composition series of $M$. Then $m=n$ and there exists $\sigma\in S_n$ such that
\[
M_i'/M_{i-1}'\cong M_{\sigma(i)}'/M_{\sigma(i)-1}'
\]
for all $1\leq i\leq n$.
\end{theorem}

%--------------------------------------------------------------------------------------------------------------------

In view of Theorem~\ref{1.1.4}, the following definitions make sense:

%--------------------------------------------------------------------------------------------------------------------

\begin{definition}
We say that a $\Lambda$-module has \textbf{finite length}\index{finite length module} if it affords a composition series
\[
(0)=M_0\subseteq M_1\subseteq \dots \subseteq M_n = M.
\]
In this case, $\ell(M):=n$ is called the \textbf{length}\index{length of a module} of $M$. Let $S$ be a simple $\Lambda$-module, then
\[
[M:S] = |\{i\in \{1,\dots,n\}\mid M_i/M_{i-1}\cong S\}|
\]
is called the \textbf{multiplicity}\index{multiplicity} of $M$ in $S$.
\end{definition}

%--------------------------------------------------------------------------------------------------------------------

\begin{definition}
A ring $\Lambda$ is called \textbf{noetherian}\index{noetherian ring} (or \textbf{artinian}\index{artinian ring}) if the regular module $\Lambda$ is noetherian (or artinian).
\end{definition}

%--------------------------------------------------------------------------------------------------------------------

\begin{example}\
\begin{enumerate}
\item Every principal ideal domain is noetherian. 
\item Let $k$ be a field and $X_1,\dots,X_n$ be indeterminates over $k$. By Hilbert's Basis Theorem, $k[X_1,\dots,X_n]$ is noetherian.
\end{enumerate}
\end{example}

%--------------------------------------------------------------------------------------------------------------------

\begin{lemma}\label{1.1.5}
Let $M$ be a finitely generated $\Lambda$-module.
\begin{enumerate}
\item If $\Lambda$ is noetherian, then $M$ is noetherian.
\item If $\Lambda$ is artinian, then $M$ is artinian.
\end{enumerate}
\end{lemma}

%--------------------------------------------------------------------------------------------------------------------

\begin{proof}
By assumption, there are $m_1,\dots,m_m\in M$ such that $M=\sum_{i\le n} \Lambda m_i$. Hence we have a surjection
\[
\pi:
\left\{
\begin{matrix}
\Lambda^n & \twoheadrightarrow & M \\
(r_1,\dots,r_n) & \mapsto & \sum_{i\le n} r_im_i
\end{matrix}
\right.
\]
If $\Lambda^n$ is an noetherian module, so are its factor modules. So it suffices to show by induction, that $\Lambda^n = \oplus_{i\le n} \Lambda$ is noetherian. The case $n=1$ is clear. For $n>1$, we have an exact sequence
\[
\begin{tikzcd}
	(0) \ar[r] & \Lambda \ar[r] & \Lambda^{n} \ar[r] & \Lambda^{n-1} \ar[r] & (0).
\end{tikzcd}
\]
By inductive hypothesis, the extreme terms are noetherian, so by Proposition~\ref{1.1.2}, the middle term is also noetherian.
\end{proof}

%--------------------------------------------------------------------------------------------------------------------

% SECTION 1.2

\section{Semisimplicity and the Jacobson radical}

%--------------------------------------------------------------------------------------------------------------------

We will study those $\Lambda$, where every short exact sequence
\[
\begin{tikzcd}
	(0) \ar[r] & M' \ar[r] & M \ar[r] & M'' \ar[r] & (0)
\end{tikzcd}
\]
splits.

%--------------------------------------------------------------------------------------------------------------------

\begin{definition}
Let $M$ be a $\Lambda$-module.
\begin{enumerate}
\item The sum of all simple submodules of $M$ is called the \textbf{socle}\index{socle} $\Soc_\Lambda(M)$ of $M$.
\item The ascending sequence $(\Soc^i_\Lambda(M))_{i\ge 0}$ which is defined inductively by
\[
\Soc_\Lambda(M):=(0)
\qq{and}
\Soc_\Lambda^i(M)/\Soc_\Lambda^{i-1}(M)\cong \Soc_\Lambda(M/\Soc_\Lambda^{i-1}(M))
\]
for all $i\ge 1$ is called the \textbf{socle series}\index{socle series} of $M$.\footnote{More precisely, let $\pi_{i-1}:M\to M/\Soc_\Lambda^{i-1}(M)$ and put $\Soc_\Lambda^i(M) := \pi_{i-1}^{-1}(\Soc_\Lambda(M/\Soc_\Lambda^{i-1}(M)))$.}
\item $M$ is called \textbf{semi-simple}\index{semi-simple module} if $\Soc_\Lambda(M)=M$.
\end{enumerate}
\end{definition}
